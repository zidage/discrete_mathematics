\documentclass{report}

\input{preamble}
\input{macros}
\input{letterfonts}

\title{\Huge{Discrete Mathematics}}
\author{\huge{Yuanfu Li}}
\date{}

\begin{document}

\maketitle
\newpage% or \cleardoublepage
% \pdfbookmark[<level>]{<title>}{<dest>}
\pdfbookmark[section]{\contentsname}{toc}
\tableofcontents
\pagebreak

\setcounter{chapter}{8}

\chapter{}
\section{Relations and Their Properties}

\subsection{Introduction}
\dfn{}{Let $A$ and $B$ be sets. A $binary\ relation\ from\ A\ to\ B$ is a subset of $A\times B$.}
We use the notation $a\boldsymbol{R}b$ to denote that $(a,b)\in \boldsymbol{R}$.

\subsection{Function as Relations}
The graph of $f$ is the set of ordered pairs $(a,b)$ such that $b=f(a)$. The graph of $f$ is a subset of $A\times B$, it is a relation from $A$ to $B$.
\nt{Conversely, if $R$ is a relation from $A$ to $B$ such that every element in $A$ is the first element of exactly one ordered pair of $R$, then a function can be defined with $R$ as its graph.}

\subsection{Relations on a Set}
\dfn{}{A $relation\ on\ a\ set\ A$ is a relation from $A$ to $A$.}
\ex{}{Let $A$ be the set $\{1,2,3,4\}$. Which ordered pairs are in the relation $R=\{(a,b)|a\ divides\ b\}.$}
\sol{Because $(a,b)$ is in $R$ if and only if $a$ and $b$ are positive integers not exceeding 4 such that $a$ divides $b$, we see that\\$R=\{(1,1),(1,2),(1,3),(1,4),(2,2),(2,4),(3,3),(4,4)\}$}
\subsection{Properties of Relations}

\dfn{}{A relation $R$ on a set $A$ is called \textit{reflexive} if $(a,a)\in R$ for every element }
\nt{We see that a relation on $A$ is reflexive if every element of $A$ is related to itself}

\dfn{}{A relation $R$ on a set $A$ is called \textit{symmetric} if $(b,a)\in R$ 
whenever $(a,b)\in R$, for all $a,b\in A$.\\A relation $R$ on a set $A$ such that for all 
$a,b\in A$, if $(a,b)\in R$ and $(b,a)\in R$, then $a=b$ is called \textit{antisymmetric}.}
In other words, a relation is symmetric if and only if $a$ is related to $b$ always implies that $b$ is related to $a$.
A relation is antisymmetric if and only if there are no pairs of distinct elements $a$ and $b$ with
$a$ related to $b$ and $b$ related to $a$. That is, the only way to have $a$ related to $b$ and $b$ related to $a$ is
for $a$ and $b$ to be the same element.

\dfn{}{A relation $R$ on a set $A$ is called \textit{transitive} is whenever $(a,b)\in R$ and
$(b,c)\in R$, then $(a,c)\in R$, for all $a,b,c\in A$.}

\subsection{Combining Relations}
\dfn{}{Let $R$ be a relation from a set $A$ to a set $B$ and $S$ a relation from $B$ to a set $C$.
The \textit{composite} of $R$ and $S$ is the relation consisting of ordered pairs $(a,c)$,
where $a\in A,c\in C$, and for which there exists an element $b\in B$ such that $(a,b)\in R$ and
$(b,c)\in S$. We denote the composite of $R$ and $S$ by $S\circ R$}
\nt{Computing the composite of two relations requires that we find elements that are the 
\textit{second} element of ordered pairs in the \textit{first} relation and the \textit{first}
element of ordered pairs in the \textit{second} relation.}
\ex{}{What is the composite of the relations $R$ and $S$, where $R$ is the relation from $\{1,2,3\}$ to
$\{1,2,3,4\}$ with $R=\{(1,1),(1,4),(2,3),(3,1),(3,4)\}$ and $S$ is the relation from 
$\{1,2,3,4\}$ to $\{0,1,2\}$ with $S=\{(1,0),(2,0),(3,1),(3,2),(4,1)\}$?}
\sol{$S\circ R=\{(1,0),(1,1),(2,1),(2,2),(3,0),(3,1)\}$}

\dfn{}{Let $R$ be a relation on the set $A$. The powers $R^n,n=1,2,3,\cdots$, are defined recursively by\\
$R^1=R$ and $R^{n+1}=R^n\circ R$.}

\thm{}{The relation $R$ on a set $A$ is transitive if and only if $R^n \subseteq R$ for $n=1,2,3,\cdots$.}
\begin{myproof}
    \par
    We first prove the "if" part of the theorem. We suppose that $R^n \subseteq R$ for $n=1,2,3,\cdots$.\\
    In particular, $R^2\subseteq R$. To see that this implies $R$ is transitive, not that if $(a,b)\in R$
    and $(b,c)\in R$, then by the definition of composition, $(a,c)\in R^2$. Because $R^2\subseteq R$, this
    means that $(a,c)\in R$. Hence, $R$ is transitive.
    \par
    Assume that $R^n\in R$, where $n$ is a positive integer. This is the inductive hypothesis. To complete the
    inductive step we must show that this implies that $R^{n+1}$ is also a subset of $R$. To show this, assume
    that $(a,b)\in R^{n+1}$. Then, because $R^{n+1}=R^n\circ R$, there is an element $x$ with $x\in A$
    such that $(a,x)\in R$ and $(x,b)\in R^n$. The inductive hypothesis, namely, that $R^n\subseteq R$,
    implies that $(x,b)\in R$. Furthermore, because $R$ is transitive, and $(a,x)\in R$ and $(x,b)\in R$, 
    it follows that $(a,b)\in R$. This shows that $R^{n+1}\subseteq R$, completing the proof.
\end{myproof}
So fucking elusive and tedious.

\section{$n$-ary Relations and Their Applications}
\subsection{Introduction}
Relations among elements from more than two sets are called $n$-ary relations.

\subsection{$n$-ary Relations}
\dfn{}{Let $A_1,A_2,\cdots,A_n$ be sets. An \textit{n-ary relation} on these sets is a subset of
$A_1\times A_2\times\cdots \times A_n$. The sets $A_1,A_2,\cdots,A_n$ are called the \textit{domains} 
of the relation, and $n$ is called its \textit{degree}.}
\ex{}{Let $R$ be the relation on $\textbf{N}\times\textbf{N}\times\textbf{N}$ consisting of triples $(a,b,c)$, where $a$, $b$ 
and $c$ are integers with $a<b<c$}

\subsection{Databases and Relations}
Understand the concept of \textbf{relational data model}, that's it.

\subsection{Operations on $n$-ary Relations}
\dfn{}{Let $R$ be an $n$-ary relation and $C$ a condition that elements in $R$ may satisfy. Then the 
\textit{selection operator} $s_C$ maps the $n$-ary relation $R$ to the $n$-ary relation of all $n$-tuples
from $R$ that satisfy the condition $C$.}

\dfn{}{The \textit{projection} $P_{i_1,i_2,\cdots,i_m}$, where $i_1<i_2<\cdots<i_m$, maps the 
$n$-tuple $(a_1,a_2,\cdots,a_n)$ to the $m$-tuple $(a_{i_1},a_{i_2},\cdots,a_{i_m})$, where $m\leq n$.}
In other words, the projection $P_{i_1,i_2,\cdots,i_m}$ deletes $n-m$ of the components of an $n$-tuple, 
leaving the $i_1$th, $i_2$th,..., and $i_m$th components.

\dfn{}{Let $R$ be a relation of degree $m$ and $S$ a relation of degree $n$. The \textit{join} 
$J_p(R,S)$, where $p\leq m$ and $p\leq n$, is a relation of degree $m+n-p$ that consists of all 
$(m+n-p)$-tuples $(a_1,a_2,\cdots,a_{m-p},c_1,c_2,\cdots,c_p,b_1,b_2,\cdots,b_{n-p})$, where the
$m$-tuple $(a_1,a_2,\cdots,a_{m-p},c_1,c_2,\cdots,c_p)$ belongs to $R$ and the $n$-tuple
$(c_1,c_2,\cdots,c_p,b_1,b_2,\cdots,b_{n-p})$ belongs to $S$.}

\dfn{}{The \textit{count} of an itemset $I$, denoted by $\sigma(I)$, in a set of 
transactions $T=\{t_1,t_2,\cdots,t_k\}$, where $k$ is a positive integer, is the
number of transactions that contain this itemset. That is, \\
$\sigma(I)=|\{t_i\in T|I\subseteq t_i\}|$.\\The \textit{support} of an itemset $T$ is the
probability that $I$ is included in a randomly selected transaction from $T$. That is,\\
$\displaystyle support(I)=\frac{\sigma(I)}{|T|}$}
The \textbf{support threshold} $s$ is specified for a particular application. 
\textbf{Frequent itemset mining} is the process of finding itemset $I$ with support
greater than or equal to $s$. Such itemsets are said to be \textbf{frequent}.

\dfn{}{If $I$ and $J$ are subsets of a set $T$ of transactions, then\\
$\displaystyle support(I\rightarrow J)=\frac{\sigma(I\cup J)}{|T|}$\\
and\\
$\displaystyle support(I\rightarrow J)=\frac{\sigma(I\cup J)}{\sigma(I)}$.}

\setcounter{section}{3}
\section{Closure of Relations}

\subsection{Introduction}
We can find all pairs of data centers that have a link by constructing a transitive relation
$S$ containing $R$ such that $S$ is a subset of every transitive relation containing $R$. 
Here, $S$ is the smallest transitive relation that contains $R$. This relation is called the
\textbf{transitive closure} of $R$.

\subsection{Different Types of Closures}

\dfn{}{If $R$ is a relation on a set $A$, then the \textbf{closure} of $R$ is with respect to $P$, 
if it exists, is the relation $S$ on $A$ with property \textbf{P} that contains $R$ and is a subset 
of every subset of $A\times A$ containing $R$ with property \textbf{P}.}

\ex{}{The relation $R=\{(1,1),(1,2),(2,1),(3,2)\}$ on the set $A=\{1,2,3\}$ is not reflexive.
By adding $(2,2)$ and $(3,3)$ to $R$ which are the only pairs of the form $(a,a)$ that are not in 
$R$, we can get a new relation which contains $R$.}
\par
The above example creates the \textbf{reflexive closure} of $R$.
\par
We see that the reflexive closure of $R$ equals $R\cup \Delta$, where $\Delta=\{(a,a)|a\in A\}$ 
is the \textbf{diagonal relation} on $A$.
\ex{}{The relation $\{(1,1),(1,2),(2,2),(2,3),(3,1),(3,2)\}$ on $\{1,2,3\}$ is not symmetric. 
By adding $(2,1)$ and $(1,3)$ to $R$ which are the only pairs of the form $(b,a)$ with $(a,b)\in R$ 
that are not in $R$, we can get a new symmetric relation contains $R$.} 
The new relation is called the \textbf{symmetric closure} of $R$.

\subsection{Path in Directed Graphs}
A path in a directed graph is obtained by traversing along edges (in the same direction as indicated by the arrow on the edge).
\dfn{}{A \textit{path} from $a$ to $b$ in the directed graph $G$ is a sequence of edges
$(x_0,x_1),(x_1,x_2),(x_2,x_3),\cdots,(x_{n-1},x_n)$ in G, where $n$ is a non-negative integer, and
$x_0=a$ and $x_n=b$, that is, a sequence of edges where the terminal vertex of an edge is the same as the 
initial vertex in the next edge in the path. This path is denoted by $x_0,x_1,x_2,\cdots,x_{n-1},x_n$
and has \textit{length} $n$. We view the empty set of edges as a path of length zero from $a$ to $a$. 
A path of length $n\geq 1$ that begins and ends at the same vertex is called a \textit{circuit} or \textit{cycle}.}

The term \textit{path} also applies to relations, Carrying over the definition from directed graphs to 
relations, there is a \textbf{path} from $a$ to $b$ in $R$ if there is a sequence of elements 
$a, x_1, x_2,\cdots, x_{n-1}, b$ with $(a,x_1)\in R, (x_1,x_2)\in R,\cdots,$ and $(x_{n-1}, b)\in R$. 
Theorem 1 can be obtained from the definition of a path in a relation.

\thm{}{Let $R$ be a relation on a set $A$. There is a path of length $n$, where $n$ is a positive integer, 
from $a$ to $b$ if and only if $(a,b)\in R^n$.}
\begin{myproof}
    We will use mathematical induction. By definition, there is a path from $a$ to $b$ of length on if 
    and only if $(a,b)\in R$, so the theorem is true when $n=1$.
    \par
    Assume that the theorem is true for the positive integer $n$. This is the inductive hypothesis. There
    is a path of length $n+1$ from $a$ to $b$ if and only if there is an element $c\in A$ such that 
    there is a path of length one from $a$ to $c$, so $(a,c)\in R$, and a path of length $n$ from $c$ to $b$, 
    that is, $(c,b)\in R^n$. Consequently, by the inductive hypothesis, there is a path of length $n+1$ from
    $a$ to $b$ if and only if there is an element $c$ with $(a,c)\in R$ and $(c,b)\in R^n$. But there is such an
    element if and only if $(a,b)\in R^{n+1}$. Therefore, there is a path of length $n+1$ from $a$ to $b$ if
    and only if $(a,b)\in R^{n+1}$. This completes the proof.
\end{myproof}

\subsection{Transitive Closures}
\dfn{}{Let $R$ be a relation on a set $A$. The \textit{connectivity relation} $R^*$ consists of 
the pairs $(a,b)$ such that there is a path of length at least one from $a$ to $b$ in $R$.}

Because $R^n$ consists of the pairs $(a,b)$ such that there is a path of length $n$ from $a$ to $b$, it
follows that $R^*$ is the union of all the sets $R^n$. In other words,
\begin{align*}
    R^*=\bigcup^\infty_{n=1} R^n
\end{align*}
The connectivity relation is useful in many models.

\ex{}{Let $R$ be the relation on the set of all subway stops in New York City that contains
$(a,b)$ if it is possible to travel from stop $a$ to stop $b$ without changing trains. What
is $R^n$ when $n$ is a positive integer? What is $R^*$?}
\sol
The relation $R^n$ contains $(a,b)$ if it is possible to travel from stop $a$ to stop $b$ by 
making at most $n-1$ changes of trains. The relation $R^*$ consists of the ordered pairs $(a,b)$ 
where it is possible to travel from stop $a$ to stop $b$ making as many changes of trains as necessary.

\thm{}{The transitive closure of a relation $R$ equals the connectivity relation $R^*$.}

\mlemma{}{Let $A$ be a set with $n$ elements, and let $R$ be a relation on $A$. If there is a 
path of length at least one in $R$ from $a$ to $b$, then there is such a path with length not 
exceeding $n$. Moreover, when $a\neq b$, if there is a path of length at least one in $R$ from
$a$ to $b$, then there is such a path with length not exceeding $n-1$.}

From Lemma 1, we see that the transitive closure of $R$ is the union of $R,R^2,R^3,\cdots,$ and
$R^n$. This follows because there is a path in $R^*$ between two vertices if and only if there is
a path between these vertices in $R^i$, for some positive integer $i$ with $i\leq n$. Because
\begin{align*}
    R^*=R\cup R^2\cup R^3\cup \cdots \cup R^n
\end{align*}
and the zero-one matrix representing a union of relations is the join of the zero-one matrices of these relations,
the zero-one matrix for the transitive closure is the join of the zero-one matrices of the first
$n$ powers of the zero-one matrix of $R$.

\thm{}{Let $\boldsymbol{M}_R$ be the zero-one matrix of the relation $R$ on a set with $n$ elements.
Then the zero-one matrix of the transitive closure $R^*$ is
\begin{align*}
    \boldsymbol{M}_{R^*}=\boldsymbol{M}_R \vee \boldsymbol{M}_R^{[2]} \vee \boldsymbol{M}_R^{[3]}
    \vee \cdots \vee \boldsymbol{M}_R^{[n]}
\end{align*}
}
\ex{}{Find the zero-one matrix of the transitive closure of the relation $R$ where
\begin{align*}
    \boldsymbol{M}_R=
    \begin{bmatrix}
        1 & 0 & 1\\
        0 & 1 & 0\\
        1 & 1 & 0
    \end{bmatrix}
\end{align*}    
}
\sol
By Theorem 3, it follows that the zero-one matrix of $R^*$ is
\begin{align*}
    \boldsymbol{M}_{R^*}=\boldsymbol{M}_R \vee \boldsymbol{M}_R^{[2]} \vee \boldsymbol{M}_R^{[3]}
\end{align*}
Because
\begin{align*}
    \boldsymbol{M}_R^{[2]}=
    \begin{bmatrix}
        1 & 1 & 1\\
        0 & 1 & 0\\
        1 & 1 & 1
    \end{bmatrix}&&
    \boldsymbol{M}_R^{[3]}=
    \begin{bmatrix}
        1 & 1 & 1\\
        0 & 1 & 0\\
        1 & 1 & 1
    \end{bmatrix}
\end{align*}
it follows that
\begin{align*}
    \boldsymbol{M}_{R^*}=
    \begin{bmatrix}
        1 & 0 & 1\\
        0 & 1 & 0\\
        1 & 1 & 0
    \end{bmatrix}\vee
    \begin{bmatrix}
        1 & 1 & 1\\
        0 & 1 & 0\\
        1 & 1 & 1
    \end{bmatrix}\vee
    \begin{bmatrix}
        1 & 1 & 1\\
        0 & 1 & 0\\
        1 & 1 & 1
    \end{bmatrix}=
    \begin{bmatrix}
        1 & 1 & 1\\
        0 & 1 & 0\\
        1 & 1 & 1
    \end{bmatrix}
\end{align*}
\subsection*{Algorithms 1}
\begin{lstlisting}[language=C++, numbers=left,
                numberstyle=\tiny,keywordstyle=\color{blue!70},
                commentstyle=\color{red!50!green!50!blue!50},frame=shadowbox,
                rulesepcolor=\color{red!20!green!20!blue!20},basicstyle=\ttfamily]
    int** transistive_closure(M = zero-one n*n matrix)
    {
        int** A = M;
        int** B = A;
        for(int i = 2; i <= n; i++)
        {
            A = A xnor M;
            B = B or A;
        }
        return B;
    }
    
\end{lstlisting}

\subsection{Warshall's Algorithm}
\dfn{}{
\par    
Suppose that $R$ is a relation on a set with $n$ elements. Let $v_1,v_2,\cdots,v_n$
be an arbitrary listing of these $n$ elements. The concept of the \textbf{interior vertices} of 
a path is used in Warshall's algorithm. If $a,x_1,x_2,\cdots,x_{m-1},b$ is a path, its interior
vertices are $x_1,x_2,\cdots,x_{m-1}$, that is, all the vertices of the path that occur somewhere
other than as the first and last vertices in the path. Note that the first vertex in the path is not
an interior vertex unless it is visited again by the path, except as the last vertex. Similarly, the last
vertex in the path is not an interior vertex unless it was visited previously by the path, except as the
first vertex.
\par
Warshall's algorithm is based on the construction of a sequence of zero-one matrices. These matrices
are $\boldsymbol{W}_0,\boldsymbol{W}_1,\cdots,\boldsymbol{W}_n$, where $\boldsymbol{W}_0=\boldsymbol{W}_R$
is the zero-one matrix of this relation, and $\boldsymbol{W}_k=[w_{ij}^{(k)}]$, where $\boldsymbol{W}_k=1$
if there is a path from $v_i$ to $v_j$ such that all the interior vertices of this path are in the
set $\{v_1,v_2,\cdots,v_k\}$(the first $k$ vertices in the list) and is 0 otherwise.
(The first and last vertices in the path may be outside the set of the first $k$ vertices in the list).
Note that $\boldsymbol{W}_n=\boldsymbol{W}_{R^*}$, because the $(i,j)$th entry of $\boldsymbol{W}_{R^*}$
is 1 if and only if there is a path from $v_i$ to $v_j$, with all interior vertices in the set
$\{v_1,v_2,\cdots,v_k\}$(but these are the only vertices in the directed graph).
}
\ex{}{Let $R$ be the relation $\{(a,d),(b,a),(b,c),(c,a),(c,d),(d,c)\}$. Let $a,b,c,d$ be a listing of
the elements of the set. Find the matrices $\boldsymbol{W}_0,\boldsymbol{W}_1,\boldsymbol{W}_2,\boldsymbol{W}_3,\boldsymbol{W}_4$
The matrix $\boldsymbol{W}_4$ is the transitive closure $R$.}
\sol
Let $v_1=a,v_2=b,v_3=c,v_4=d$.$\boldsymbol{W}_0$ is the matrix of the relation. Hence,
\begin{align*}
    \boldsymbol{W}_0=
    \begin{bmatrix}
        0 & 0 & 0 & 1\\
        1 & 0 & 1 & 0\\
        1 & 0 & 0 & 1\\
        0 & 0 & 1 & 0
    \end{bmatrix}\ 
    \boldsymbol{W}_1=
    \begin{bmatrix}
        0 & 0 & 0 & 1\\
        1 & 0 & 1 & 1\\
        1 & 0 & 0 & 1\\
        0 & 0 & 1 & 0
    \end{bmatrix}\ 
    \boldsymbol{W}_2=\boldsymbol{W}_3=
    \begin{bmatrix}
        0 & 0 & 0 & 1\\
        1 & 0 & 1 & 1\\
        1 & 0 & 0 & 1\\
        1 & 0 & 1 & 1
    \end{bmatrix}\ 
    \boldsymbol{W}_4=
    \begin{bmatrix}
        1 & 0 & 1 & 1\\
        1 & 0 & 1 & 1\\
        1 & 0 & 1 & 1\\
        1 & 0 & 1 & 1
    \end{bmatrix}
\end{align*}

\mlemma{}{Let $\boldsymbol{W}_k=[w_{ij}^{[k-1]}]$ be the zero-one matrix that has a 1 in its $(i,j)$th
position if and only if there is a path from $v_i$ to $v_j$ with interior vertices from the set $\{v_1,v_2,\cdots,v_k\}$. Then
\begin{align*}
    w_{ij}^{[k]}=w_{ij}^{[k-1]}\vee (w_{ik}^{[k-1]}\wedge w_{kj}^{[k-1]})\
\end{align*}
}

\subsection*{Algorithm 2}
\begin{lstlisting}[language=C++, numbers=left,
    numberstyle=\tiny,keywordstyle=\color{blue!70},
    commentstyle=\color{red!50!green!50!blue!50},frame=shadowbox,
    rulesepcolor=\color{red!20!green!20!blue!20},basicstyle=\ttfamily]
int** Warshall(M = n*n zero-one matrix)
{
    W = M;
    /* Suppose the first index of an array is 1. */
    for (int k = 1; k <= n; k++)
    {
        for (int i = 1; i <= n; i++)
        {
            for (int j = 1; j <= n; j++)
            W[i][j] = W[i][j] | (W[i][k] & W[k][j]) 
        }
    }
    return W;
}

\end{lstlisting}

\subsection{Equivalence Relations}
\dfn{}{A relation on a set $A$ is called an \textit{equivalence relation} if it is reflexive, 
symmetric, and transitive.}
\dfn{}{Two elements $a$ and $b$ that are related by an equivalence relation are called \textit{equivalent}.
The notation $a\sim b$ is often used to denote that $a$ and $b$ are equivalent elements with respect to a 
particular equivalence relation.}

\dfn{}{Let $R$ be an equivalence relation on a set $A$. The set of all elements that are related to 
an element $a$ of $A$ is called the \textit{equivalence class} of $a$. The equivalence class of $a$ with
respect to $R$ is  denoted by $[a]_R$. When only one relation is under consideration, we can delete the subscript
$R$ and write $[a]$ for this equivalence class.}

\subsection{Equivalence Classes and Partitions}
$R$ is an equivalence relation. We can see that $R$ splits all students in $A$ in to a collection of 
disjoint subsets, where each subset contains students with a specified major. Furthermore, these subsets
are equivalence classes of $R$. This example illustrates how the equivalence classes of an equivalence 
relation partition a set into disjoint, non-empty subsets.
\thm{}{Let $R$ be an equivalence relation on a set $A$. These statements for elements $a$ and $b$ of $A$
are equivalent: 
\begin{align*}
    aRb&&[a]=[b]&&[a]\cap [b]\neq \emptyset
\end{align*}
}

Now we are going to show how an equivalence relation \textit{partitions} a set. Let $R$ be an
equivalence relation on a set $A$. The union of the equivalence classes of $R$ is all of $A$, 
because an element $a$ of $A$ is in its own equivalence class, namely, $[a]_R$. In other words,
\begin{align*}
    \bigcup_{a\in A}[a]_R=A.
\end{align*}
In addition, from Theorem 9.4.4, it follows that these equivalence classes are either equal or disjoint, so
\begin{align*}
    [a]_R\cap [b]_R=\emptyset
\end{align*}
when $ [a]_R\neq [b]_R$
\par
These two observations show that the equivalence classes form a partition of $A$, because they
split $A$ into disjoint subsets. More precisely, a \textbf{partition} of a set $S$ is a collection of
disjoint non-empty subsets of $S$ that have $S$ as their union. In other words, the collection of
subsets $A_i,i\in I$ forms a partition of $S$ is and only if
\begin{align*}
    &A_i\neq \emptyset\ for\ i\in I,\\
    &A_i\cap A_j = \emptyset\ when\ i\neq j,
\end{align*}
and
\begin{align*}
    \bigcup_{i\in I}A_i=S
\end{align*}

\thm{}{Let $R$ be an equivalence relation on a set $S$. Then the equivalence classes of $R$ form a 
partition of $S$. Conversely, given a partition$\{A_i | i\in I\}$ of the set $S$, there is an
equivalence relation $R$ that has the sets $A_i,i\in I$, as its equivalence classes.}
\end{document}
